\documentclass[journal]{IEEEtai}

\usepackage[colorlinks,urlcolor=blue,linkcolor=blue,citecolor=blue]{hyperref}

\usepackage{color,array}

\usepackage{graphicx}

\newtheorem{theorem}{Theorem}
\newtheorem{lemma}{Lemma}
\setcounter{page}{1}

\begin{document}

\title{Navigating Ethical Challenges in AI-Driven Decision-Making (June 2024)} 

\author{Rohan Raj - 00152870
\thanks{This research proposal was submitted on 19th June 2024.}

\thanks{Rohan Raj is with the Computer Science and Artificial Intelligence Department, Technische Hochschule, Ingolstadt, CO 85049 Germany (e-mail: ror0011@thi.de).}}

\markboth{Journal on Artificial Intelligence, Vol. 00, No. 0, June 2024}
{Rohan Raj \MakeLowercase{\textit{et al.}}}

\maketitle
\begin{abstract}

This research proposal explores the ethical challenges that come with the deployment of artificial intelligence (AI) in automated decision-making systems. Nowadays, AI is being used extensively in various sectors such as finance, healthcare, and criminal justice. While AI can improve efficiency and reduce human biases, it also presents significant challenges. Unlike human decision-makers, AI systems often lack transparency in their decision-making processes, making accountability difficult. Moreover, these systems can perpetuate or even exacerbate existing biases due to the data they are trained on. The lack of transparency and accountability in AI systems raises serious ethical concerns, as stakeholders cannot easily understand or contest AI decisions. Therefore, this proposal aims to systematically identify and address these ethical challenges. By focusing on mitigating bias, enhancing transparency, and establishing robust accountability mechanisms, this study seeks to ensure the responsible and ethical deployment of AI in decision-making systems, contributing to the development of AI systems that are fair, transparent, and accountable, fostering greater trust and acceptance of AI technologies across various sectors.
\end{abstract}

\begin{IEEEImpStatement}

Artificial intelligence (AI) is increasingly used in sectors such as finance, healthcare, and criminal justice to improve efficiency and decision-making. However, the ethical challenges associated with AI, such as bias, lack of transparency, and accountability, can undermine its benefits and lead to mistrust. Our research aims to develop a comprehensive framework to address these ethical issues. By mitigating bias, enhancing transparency, and establishing robust accountability mechanisms, we strive to foster greater trust and acceptance of AI technologies. This will significantly impact the implementation of AI systems and promote ethical practices across various industries. Our work seeks to ensure that AI is deployed responsibly, paving the way for more equitable and trustworthy AI applications, enhancing their societal, while minimizing ethical risks.

\end{IEEEImpStatement}

\begin{IEEEkeywords}
AI, accountability, bias, decision-making, ethics, transparency.

\end{IEEEkeywords}

\section{Introduction}

\IEEEPARstart{W}{ith} the growing use of technology, Artificial Intelligence (AI) is gaining rapid popularity. It is used extensively in many different sectors such as finance, healthcare and criminal justice, transforming decision-making processes by improving efficiency and reducing human biases. In finance, AI algorithms analyze vast amounts of data to predict stock market, make investment decisions and detect fraudulent activities. In healthcare, AI assists in diagnosing different diseases and personalizing treatment plans. In criminal justice, AI systems are used to identify and predict areas with high crime, ensuring safety of humans, and assist in courts, which include making decisions about sentences. These applications show how immense the potential of AI is in transforming these industries and enhancing outcomes.

\section{Importance of Ethical AI}

The importance of AI can seen from its extensive use in various sectors. However, it is equally important that it is being used ethically. As it does not function exactly like a human, it cannot distinguish between right and wrong, and good and bad. So it is important to configure it in a way that it does distinguish.

It can be non-ethical in several ways, such as concerns arise regarding biases inherent in AI systems due to the data they are trained on [1], which may lead to unfair and discriminatory outcomes. Moreover, the opacity of AI decision-making processes poses challenges to transparency [2], which makes it difficult for people to understand and trust it. If the system involves machine learning, it will typically be opaque even to the expert, who will not know how a particular pattern was identified, or even what the pattern is [2].

% Democratising Decisions About Technology found that participants were supportive of the use of AI within the NHS to assist doctors and nurses, but were more sceptical about its use in criminal justice, where the dangers of bias or lack of human oversight weighed heavier [3].

There is also a question of how to hold AI accountable if we were to hold it accountable. Therefore, addressing these challenges is crucial for its responsible development and deployment. We need to make sure that its benefits are realized and that we are not compromising ethical standards.

\section{Current State of Art}

AI is currently making significant progress in image
recognition, computer vision, language manipulation, or prediction, with huge possible impacts for healthcare, transportation, media, or the military [1].

In healthcare for example, deep learning algorithms of image recognition are able to perform at human or super-human levels in a variety of tasks. For example, a new model called MENDDL (Multinode Evolutionary Neural Networks for Deep Learning) is faster than human at finding defects in electron microscopy (US Department of Energy 2018). Another example includes DeepMind,
whose software can identify 50 eye diseases as accurately as human doctors, by looking at 3D scans of retinas (Vincent 2018).

While progress in image recognition and computer vision already delivers breakthrough applications with a huge potential for positive impact, there is
also a darker side to this technology. The technique called
deepfake superimposes existing images or videos onto source images or videos, using a machine learning procedure called generative adversarial
network (GAN), which makes it possible to
create fake videos of anyone saying anything. It opens up the possibility of being personally targeted
as subjects of fake videos, with catastrophic consequences for one’s reputation [1].

\section{Research Gap}

Despite the significant progress in AI technologies, there remains a critical gap in understanding and addressing the ethical challenges associated with with it. While there are concerns about bias, transparency, and accountability [2], comprehensive frameworks and solutions to mitigate these challenges are still underdeveloped, necessitating further research and exploration.

\section{Research Questions}

To address the identified research gap, this proposal seeks to answer the following questions:
\begin{enumerate}
    \item How can biases in AI systems be identified and mitigated to ensure fair decision-making?
    \item What strategies can enhance transparency in AI algorithms to improve understanding and trust?
    \item How can accountability mechanisms be effectively integrated into AI systems to uphold ethical standards?
\end{enumerate}

\section{Proposed Solution/Experiment}

The proposed research will develop a comprehensive framework aimed at enhancing the ethical deployment of AI in decision-making systems. This framework will involve:
\begin{itemize}
    \item Analyzing existing AI algorithms to identify biases and develop mitigation strategies.
    \item Implementing transparency measures to elucidate AI decision-making processes.
    \item Designing and evaluating accountability mechanisms to ensure responsible use of AI.
\end{itemize}

\section*{References}

Dorobantu, Marius. (2019). Recent advances in Artificial Intelligence (AI) and some of the issues in the theology and AI dialogue.

Floridi and Taddeo, "2.3 Opacity of AI Systems," in *The Stanford Encyclopedia of Philosophy*, Stanford University, 2016.

“FORUM FOR ETHICAL AI.” RSA Journal, vol. 165, no. 4 (5580), 2019, pp. 6–9. JSTOR, https://www.jstor.org/stable/26936883. Accessed 19 June 2024.

\begin{IEEEbiography}{Rohan Raj}{\space}was born in Karachi, Sindh, Pakistan in 2002. He is currently pursuing B.S. degree in computer science and artificial intelligence at Technische Hochschule, Ingolstadt, Germany since 2023.

Since 2024, he is working as a Software Developer at fortiss, Munich, Germany. His work includes working on application-oriented research project EILE, focusing on implementing an energy management system to create CO2 balances for production process steps. His tasks involve development, testing, and maintenance of software bundles and widgets, along with integration of external forecasting services and data analytics.

\end{IEEEbiography}

\end{document}
